\documentclass[a4paper,10pt]{article}
\usepackage[utf8]{inputenc}
\usepackage{listings}
% Er zijn talloze parameters ...
\lstset{language=C++, showstringspaces=false, basicstyle=\small,
  numbers=left, numberstyle=\tiny, numberfirstline=false,
  stepnumber=1, tabsize=8, 
  commentstyle=\ttfamily, identifierstyle=\ttfamily,
  stringstyle=\itshape}
\usepackage{amsmath}
  
%opening
\title{ Hogebomen }
\author{ Lisette de Schipper (s1396250) en Micky Faas (s1407937) }
\date{}

\begin{document}

\maketitle

\section*{Inleiding}
Al vanaf de middelbare school moeten we afgeleiden nemen van functies met een of meerdere onbekenden.
Nu hebben wij een progrAmma geschreven dat dit werk van de gebruiker overneemt.
De gebruiker vult simpelweg een expressie in prefix notatie en het programma doet vervolgens al het werk door middel van een expressieboom.

\section*{Werkwijze}
De broncode van het programma bestaat uit de volgende bestanden:
\begin{itemize}
 \item ExpressionAtom.cc
 \item ExoressionAtom.h
 \item ExpressionTree.cc
 \item ExpressionTree.h
 \item main.cc
 \item main2.cc
 \item Tree.h
 \item TreeNode.h
 \item TreeNodeIterator.h
\end{itemize}

\noindent In de terminal kun je in \texttt{hogebomen} ``make'' typen en vervolgens naar de \texttt{bin}-directory gaan, om daar \texttt{./hogebomen} te runnen.

\subsection*{class TreeNode}
Hier staan alle knopen gedefinieerd.
Van elke knoop leggen we het volgende vast:
\begin{itemize}
 \item inhoud
 \item wie de ouder is, en wie de kinderen zijn
\end{itemize}
Daarnaast zijn er nog een aantal functies gedefinieerd, die allemaal voor zich spreken.

\subsection*{class Tree}
De meeste functies hier spreken voor zich. \\
Als er een kind toegevoegd moet worden aan een knoop (\texttt{insert()}), maar die knoop is al vol,
zijn er drie mogelijke reacties door het programma. Dit geldt ook voor het vervangen van een knoop (\texttt{replace()}). Dit gedrag kan als argument aan deze twee functies worden meegegeven.

\begin{itemize}
\item ABORT\_ON\_EXISTING, het programma wordt afgebroken en 0 wordt geretourneerd.
\item MOVE\_EXISTING, maak van de ouders kind een kind van de nieuwe knoop
\item DELETE\_EXISTING, verwijder een van de kinderen
\end{itemize}

\noindent Tot slot is er de functie \texttt{pushBack()} die een knoop toevoegt op de eerste volgende plek (gaat uit van een volle binaire boom, van links naar rechts gevuld).

\subsection*{class TreeNodeIterator}
Dit bestand coördineert de drie mogelijke wandelingen door de boom: in-order wandeling, pre-order wandeling en post-order wandeling.
In onze implementatie kun je erdoorheen lopen door gebruik te maken van een iterator.
Een voorbeeld van het gebruik van zo'n iterator is als volgt:
\begin{verbatim}
Tree<char> tree;

//wat waardes om de boom te vullen
    tree.pushBack( 'a' );
    tree.pushBack( 'b' );
    tree.pushBack( 'c' );
    tree.pushBack( 'd' );

// in-orde wandeling
Tree<char>::iterator_in it( tree.begin_in( ) );

for( ; it !=tree.end_in( ); ++it ) {
    cout << *it << " ";
}

// of de (simpelere) standaard pre-orde
// dmv de begin() en end() functies (C++11)
for( auto c : tree )
    cout << c << endl;
    
\end{verbatim}
De (abstracte) klasse (\texttt{TreeNodeIterator}) is een klasse die gebruikt wordt door de klassen die deze overerven/specialiseren.
Deze klasse bevat dus een aantal algemene functies.
Zo wordt hier de ``++it'' uit het voorbeeld hierboven gedefineerd, deze roept in de kind-klassen de virtuele functie (\texttt{next()}) aan om de volgende stap te zetten.
De operatie ``++'' kunnen we dus ook gebruiken bij de andere 2 wandelingen.

De volgende 3 klassen \texttt{TreeNodeIterator\_in}, \texttt{TreeNodeIterator\_pre} en \texttt{TreeNodeIterator\_post} bevatten
allemaal de logica voor de bijbehorende wandelingen. Deze wandelingen zijn vrij simpel zonder stack geimplementeerd omdat we ervoor hebben gekozen om elke \texttt{TreeNode} een pointer naar zijn ouder te geven.

\subsection*{class ExpressionAtom}
In ExpressionAtom staan alle inhouden/typen van knopen gedefinieerd die er gebruikt kunnen worden en ook een aantal operaties die we op ze kunnen uitvoeren.
Zo worden $==$, $+$, $-$, $*$, $/$, $<$,$>$,$<=$ en $>=$ ge-overload zodat ze door deze atomen gebruikt kunnen worden.

\noindent De types die door ExpressionAtom gebruikt kunnen worden:
\begin{itemize}
 \item Integer
 \item Float/Kommagetal
 \item Breuk (aparte struct \texttt{Fraction})
 \item Variabele
 \item Operator
    \begin{itemize}
      \item +
      \item -
      \item *
      \item /
      \item \^\
    \end{itemize}
 \item Functie
  \begin{itemize}
    \item sin
    \item cos
    \item tan
    \item log
    \item ln
    \item wortel
    \item absolute waardes
    \item e
    \item pi
    \item unaire -
  \end{itemize}
\end{itemize}

\subsection*{class ExpressionTree}
Om strings te ontleden naar bomen maken we gebruik van een tokenizer (\texttt{tokenize()}) en een parser (\texttt{fromString()}).
De tokenizer zet elk element om in een token van type \texttt{ExpressionAtom}.
Al die tokens zijn de input van de parser die ze verder verwerkt tot een boom. De uitkomst is altijd ondubbelzinnig omdat op voorhand de ariteit van elke token bekend is (dmv van \texttt{ExpressionAtom::arity()}).
De \texttt{differentiate()}, \texttt{generateInOrder)(} en \texttt{simplify()}-functies pakken hun problemen recursief aan.

Bij het evalueren worden alle variabelen door conrete waarden vervangen door de functies \texttt{mapVariables()} en/of \texttt{mapVariable()}.
Het resultaat wordt hierna versimpeld in \texttt{simplify()}
\subsection*{hogebomen}
Simpele interface voor de gebruiker om te differentiëren, evalueren, simplificeren en converteren naar dot-notatie. Bronbestand \texttt{main.cc} wordt gebouwd naar \texttt{bin/hogebomen}

\subsection*{hogebomen2}
Klein testprogramma. Als je deze code uitvoerd, zul je zien dat de 3 wandelingen (pre, post en in) goed uitgevoerd worden. Bronbestand \texttt{main2.cc} wordt gebouwd naar \texttt{bin/hogebomen2}


\section*{Voorbeelden}
\subsection*{Differentiëren}
Differentieerbare functies:
\begin{itemize}
 \item $tan(f(x))$
 \item $sin(f(x))$
 \item $cos(f(x))$
 \item $ln(f(x))$
 \item $sqrt(f(x))$
 \item $log(f(x))$
 \item $abs(f(x))$
 \item $f(x) + g(x)$
 \item $f(x) - g(x)$
 \item $f(x) / g(x)$
 \item $f(x) * g(x)$
 \item $f(x) ^{g(x)}$
  \begin{itemize}
  \item $x ^{n}$
  \item $x ^{f(x)}$
  \item $e ^{f(x)}$
  \item $n ^{f(x)}$
  \end{itemize}
\end{itemize}
en combinaties hiervan. \\

\noindent Nu een aantal voorbeelden. De expressies zijn gegeven in de infix notatie en hun afgeleides ook.
Alle afgeleides zijn correct.

\subsubsection*{tan(ax)}
Oorspronkelijke expressie: $tan(x ^ -7)$ \\
Uitkomst: $((cos(x^{-7})*((-7/(x^8))*cos(x^{-7})))-(sin(x^{-7})*((-(-7/(x^8)))*sin(x^{-7}))))/(cos(x^{-7})^2)$

\subsubsection*{sin(ax)}
Oorspronkelijke expressie: $sin(x * 4)$ \\
Uitkomst: $4cos(x*4)$

\subsubsection*{cos(ax)}
Oorspronkelijke expressie: $cos(x / * t ^ 3 x)$ \\
Uitkomst: $(-((((t^3)*x)-(x*((x*)+(t^3))))/(((t^3)*x)^2)))*sin(x/((t^3)*x))$

\subsubsection*{ln(ax)}
Oorspronkelijke expressie: $ln(4 * x + x)$ \\
Uitkomst: $5/((4x)+x)$

\subsubsection*{sqrt(ax)}
Oorspronkelijke expressie: $sqrt(ln(12.3 * x ^ 2))$ \\
Uitkomst: $((12.3(2x))/(12.3(x^2)))/(2sqrtln(12.3(x^2)))$

\subsubsection*{log(ax)}
Oorspronkelijke expressie: $^{4}log(5x)$ \\
Uitkomst: $(1/ln(4))*(5/(5x))$

\subsubsection*{abs(ax)}
Oorspronkelijke expressie: $abs(cos( 3 * x))$ \\
Uitkomst: $(cos(3x)*(-3sin(3x)))/abscos(3x)$

\subsubsection*{f(x) + g(x)}
Oorspronkelijke expressie: $ln(4x) + (4/( x^2))$ \\
Uitkomst: $(4/(4x))+((0-(4(2x)))/((x^2)^2))$

\subsubsection*{f(x) / g(x)}
Oorspronkelijke expressie: $3x/(2^x)$ \\
Uitkomst: $(((2^x)*3)-((3x)*(ln(2)*(2^x))))/((2^x)^2)$

\subsubsection*{f(x) * g(x)}
Oorspronkelijke expressie: $x * 3$ \\
Uitkomst: $3$

\subsubsection*{f(x) \^\ g(x)}
Oorspronkelijke expressie: $x ^ x$ \\
Uitkomst: $((x*(1/x))+ln(x))*(e^{(ln(x)*x)})$

\subsection*{Wandelingen}
main2.cc bevat de volgende boom:
\begin{verbatim}
           =
    *              /
+     -        :      %
1 2   3 4      5 6    7 8
\end{verbatim}


\noindent Output van main2.cc: \\

\noindent in-order traversal: 1 + 2 * 3 - 4 = 5 : 6 / 7 \% 8 \\ 
post-order traversal: 1 2 + 3 4 - * 5 6 : 7 8 \% / = \\
pre-order traversal: = * + 1 2 - 3 4 / : 5 6 \% 7 8

\section*{Appendix}

\subsection*{ExpressionAtom.h}
\lstinputlisting{../src/ExpressionAtom.h}
\subsection*{ExpressionAtom.cc}
\lstinputlisting{../src/ExpressionAtom.cc}
\subsection*{ExpressionTree.h}
\lstinputlisting{../src/ExpressionTree.h}
\subsection*{ExpressionTree.cc}
\lstinputlisting{../src/ExpressionTree.cc}
\subsection*{main.cc}
\lstinputlisting{../src/main.cc}
\subsection*{Tree.h}
\lstinputlisting{../src/Tree.h}
\subsection*{TreeNode.h}
\lstinputlisting{../src/TreeNode.h}
\subsection*{TreeNodeIterator.h}
\lstinputlisting{../src/TreeNodeIterator.h}

\end{document}
