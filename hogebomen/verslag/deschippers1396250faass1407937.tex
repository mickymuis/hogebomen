\documentclass[a4paper,10pt]{article}
\usepackage[utf8]{inputenc}
\usepackage{listings}
% Er zijn talloze parameters ...
\lstset{language=C++, showstringspaces=false, basicstyle=\small,
  numbers=left, numberstyle=\tiny, numberfirstline=false,
  stepnumber=1, tabsize=8, 
  commentstyle=\ttfamily, identifierstyle=\ttfamily,
  stringstyle=\itshape}
\usepackage{amsmath}
  
%opening
\title{ Hele Hogebomen }
\author{ Lisette de Schipper (s1396250) en Micky Faas (s1407937) }
\date{}

\begin{document}

\maketitle

\section*{Inleiding}


\section*{Werkwijze}

\section*{Experimenten}

Hieronder zullen we kort enkele experimenten toelichten die wij hebben uitgevoerd. In het volgende hoofdstuk staan vervolgens de resultaten beschreven.

\subsection*{Hooiberg}

``Hooiberg'' is de naam van het testprogramma dat we hebben geschreven speciaal ten behoeven van onze experimenten.
Het is een klein console programma dat woorden uit een bestand omzet tot een boom in het geheugen. 
Deze boom kan vervolgens worden doorzocht met de input uit een ander bestand de ``naalden''.
De syntax is alsvolgt:
\begin{verbatim}
hooiberg type hooiberg.txt naalden.txt [treap-random-range]
\end{verbatim}
Hierbij is \texttt{type} \'e\'en van \texttt{bst, avl, splay, treap}, het eerste bestand bevat de invoer voor de boom, het tweede bestand een set strings als 

\section*{Appendix}

\subsection*{main.cc}
\lstinputlisting{../src/main.cc}
\subsection*{hooiberg.cc}
\lstinputlisting{../src/hooiberg.cc}
\subsection*{Tree.h}
\lstinputlisting{../src/Tree.h}
\subsection*{TreeNode.h}
\lstinputlisting{../src/TreeNode.h}
\subsection*{TreeNodeIterator.h}
\lstinputlisting{../src/TreeNodeIterator.h}

\end{document}
